\section{Conclusiones}

Podemos ver que caracterizar instrumentos y separar audio presesntan un desafio complejo para el cual existen una gamma de t\'ecnicas muy diversas para resolverlo. Ademas, la mayoria suele tener resultados parcialmente exitosos, de manera que resulta dificil resolver el problema de manera exacta y la manera de encararlo resulta ser estimando una soluci\'on relativamente buena.

En particular, la factorizaci\'on NMF nos fue muy util para descomponer el sonido en una cantidad arbitraria de componentes distintos. Esta factorizaci\'on es usada en distintas tecnicas de manipulaci\'on de audio pero usualmente resulta ser una pequena parte de un proceso mas grande para separar el audio.

Podemos ver que caracterizar instrumentos de manera "manual" en el sentido de intentar usar una tecnica particular resulta ser dificil para caracterizarlos, mas que nada debido a la cantidad de variables distintas a tener en cuenta sobre la senal, esto nos lleva a pensar que una solucion de ML puede llegar a ser la mas razonable para este tipo de problemas. Aun m\'as, nuestros resultados muestran como un modelo relativamente simple tiene una performance masomenos aceptable sobre un conjunto de datos, dejando abierta la pregunta de si se puede llegar a usar nuestro modelo como un modulo individual de un modelo mas grande.

Con respecto a la separaci\'on de audios, en nuestra experimentaci\'on logramos obtener casos donde dicha factorizaci\'on nos permitio separar alguna caracteristica del audio y relacionarla con el original, dando la idea de que este m\'etodo podr\'ia llegar a ser util como un modulo de un modelo mas grande, al igual que el modelo de ML.